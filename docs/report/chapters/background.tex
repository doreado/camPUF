% In the background chapter you should provide all the information required to acquire a sufficient knowledge to understand other chapters of the report. Suppose the reader is not familiar with the topic; so, for instance, if your project was focused on implementing a VPN, explain what it is and how it works. This chapter is supposed to work kind of like a "State of the Art" chapter of a thesis.\\ Organize the chapter in multiple sections and subsections depending on how much background information you want to include. It does not make any sense to mix background information about several topics, so you can split the topics in multple sections.\\Assume that the reader does not know anything about the topics and the tecnologies, so include in this chapter all the relevant information. Despite this, we are not asking you to write 20 pages in this chapter. Half a page, a page, or 2 pages (if you have a lot of information) for each `topic`(i.e. FreeRTOS, the SEcube, VPNs, Cryptomator, PUFs, Threat Monitoring....thinking about some of the projects...).
\chapter{Background}
In this era where digital security is very important, hardware-based security  is on the rise because provide very complex and reliable solutions. Physically Unclonable Functions (PUFs) have emerged as a promising approach to address the challenges of hardware authentication, secure key generation, and anti-counterfeiting measures. Among the various PUF implementations, Complementary Metal-Oxide-Semiconductor Arbiter PUFs (CAMPUFs) stand out as a powerful hardware security primitive based on CMOS technology.

CAMPUFs leverage the unique physical variations inherent in CMOS integrated circuits to generate unpredictable and practically unclonable responses. The foundation of CMOS technology, with its low power consumption and high integration capabilities, makes it an ideal platform for building complex digital circuits and implementing PUFs.

This documentation will present and underly principles of PUFs, explain the workings of CAMPUFs.
\section{General Concepts}
\begin{itemize}
\item \textbf{PUF}: Physically Unclonable Function (PUF) is a security metric that exploits inherent device physical variations to produce an unclonable, 
unique device response to a given input. Unclonability means that each PUF has a unpredictable response to stimuli, because the response is created by complex interactions between many random components.
PUFs implement a challenge-response type of authentication. They act as a unique device identifier.
//(TO WRITE: EXAMPLES OF PUFs)
\item \textbf{Fixed Pattern Noise}: it is a particular noise pattern present on digital imaging sensors, caused by small differences in specific sensor pixels, patterns can result in brighter or darker pixels.
    It is created from two sources: PRNU (Photo Response Non-Uniformity) and DSNU (Dark Signal Non-Uniformity). The latter will be the focus of our PUF implementation.
\item \textbf{PRNU}: 

\item \textbf{DSNU}: Dark Signal Non-Uniformity is a type of noise that occurs in dark images. The intensity values for pixels does not start from zero, an offset is added on every pixel,
    to avoid signal values dropping lower than zero. This makes each pixel to always have a non-zero value, which is the \textit{bias}.
    Fluctuation in the bias is the DSNU
\item \textbf{Challenge-Response Authentication}: authentication method based on a unique challenge provided to a device by an authenticator.
    The device must create a response to the challenge, based on unique physical properties that only that specific device has.
\item 

\end{itemize}

Physically unclonable functions (PUFs) are a technique in hardware security that exploits inherent device variations to produce an unclonable, unique device response to a given input.
\subsection{Purposes} TOTALMENTE COPIATI DA CHATGPT QUESTI

\textbf{Key Generation}: PUFs can generate cryptographic keys directly from the unique physical responses of the ICs. These keys can be used for secure communication, encryption, and data protection.

\textbf{Challenge-Response Mechanism}: PUFs operate on a challenge-response mechanism, where a unique challenge is provided to the device, and the corresponding response is generated based on the unique physical characteristics of that particular IC.

\textbf{Randomness and Entropy Generation}: PUFs can serve as a high-quality source of randomness and entropy, critical for cryptographic protocols and secure communication.

\textbf{Authentication}: PUF responses can be used for secure device authentication, verifying the identity and integrity of hardware components in various applications, such as secure booting and secure firmware updates.



\textbf{Uniqueness and Unpredictability}: PUFs generate device-specific responses based on the inherent physical variations during manufacturing. As a result, each instance of the IC exhibits a unique response, making it practically impossible to clone or replicate the device.

\textbf{Anti-Counterfeiting Measures}: PUFs play a crucial role in preventing counterfeiting and unauthorized duplication of hardware components, as the uniqueness of the responses ensures the authenticity of genuine devices.

\subsection{Security Properties and Applications of PUFs}
\textbf{Key Security Properties of PUFs}


\textbf{Unpredictability}: The inherent randomness linked to  PUF responses, even when given the same challenge multiple times. This property makes stonger the security of PUF-based authentication and key generation.


\textbf{Uniqueness and Unclonability}: PUFs depend on the uniqueness of their physical microstructure. This microstructure depends on random physical factors introduced during manufacturing that gives unique reponses. These factors are unpredictable and uncontrollable, which makes it virtually impossible to duplicate or clone the structure.


\textbf{Resistance to Physical Attacks}: PUFs are resilient against various physical attacks, including invasive attacks like reverse engineering and probing, as well as non-invasive attacks like side-channel analysis.


\textbf{Practical Applications of PUFs}


\textbf{Secure Key Generation}: PUF responses can be utilized to generate cryptographic keys without the need for additional storage of secret keys, enhancing the security of cryptographic protocols.


\textbf{Hardware Authentication}: PUFs can be employed for secure device authentication, ensuring the legitimacy of hardware components and protecting against unauthorized access.


\textbf{Anti-Counterfeiting Measures}: PUFs serve as a powerful tool for detecting counterfeit devices, as the unique responses enable the verification of genuine products.


\textbf{Secure Communication}: PUF-based keys are valuable for securing communication channels, enabling secure and authenticated data transmission.


\subsection{Categories of Implementation}
PUFs can be categorized based on different operational principles and sources of randomness
\subsubsection{Intrinsic PUFs}

Intrinsic PUFs are characterized by their reliance on the inherent physical variations present within a single chip during the semiconductor manufacturing process. These variations arise due to manufacturing imperfections, process fluctuations, and random dopant fluctuations. The unique characteristics of each individual chip create a fingerprint-like response, making Intrinsic PUFs ideal for hardware authentication and identification purposes.

Common Implementations of Intrinsic PUFs are:
\begin{itemize}
\item Delay PUFs: Delay PUFs exploit the differences in signal propagation delay along various paths within the circuit. By measuring these delays, a set of unique challenge-response pairs can be generated, forming the basis for secure authentication.

\item Ring Oscillator PUFs: Ring oscillator PUFs use the frequency differences in ring oscillators, circuits comprising an odd number of inverters. The varying delays in these oscillators result in distinctive response patterns, enabling the generation of cryptographic keys and unique device identifiers.

\item SRAM PUFs: SRAM PUFs exploit the randomness in the power-up behavior of standard static random-access memory on a chip as a PUF.

\item VIA PUFs: the Via PUFs technologies are based on "via" or "contact" formation during the standard CMOS fabrication process. 
\end{itemize}
\subsubsection{Extrinsic PUFs}

Extrinsic PUFs differ from Intrinsic PUFs as they derive their responses from external environmental factors that influence the behavior of the chip. These external factors may include temperature variations, light levels, power supply fluctuations, and electromagnetic interference. The responses generated by Extrinsic PUFs can change under different operating conditions, leading to additional sources of randomness.

Common Implementations of Extrinsic PUFs are:
\begin{itemize}
\item Ambient Light PUFs: Ambient light PUFs utilize the incident light level on the chip's surface to create distinct response patterns. Variations in light intensity cause corresponding variations in the generated responses, enabling unique identification.

\item Temperature PUFs: Temperature PUFs exploit temperature-induced changes in the electrical characteristics of the chip. Different temperature levels result in varied responses, contributing to the unclonable behavior of the device.
\end{itemize}
\section{\textbf{Technical Manual}}
(Aimed at engineers, technical users that work directly with source code etc.)

(Provide architectural overview of the project components and explain their interactions)

(code structure, key modules + tools required to build project and run it)

This camPUF implementation is based on CMOS image sensors, typically present on modern smartphone cameras.
The entirety of the project consists of a digital signal processing part and a device authentication part.
All the project is implemented using Python3
Here is the list of the libraries used:

\begin{itemize}
\item Opencv: computer vision library, used to extract data from raw images
\item scipy: python library that provides mathematical instruments and algorithms
\item MatPlotlib: graph library for math visualizations
\end{itemize}

There are two main python modules, one for DSNU extraction and one for the enrollment procedure.
%The modules are \textit{extract_dsnu.py} and \textit{enrollment.py}
%\textit{server.py}
