\chapter{Introduction}
% DELETE THE TEXT BELOW
%In this first chapter we expect you to introduce the project explaining what the project is about, what is the final goal, what are the topics tackled by the project, etc.\newline The introduction must not include any low-level detail about the project, avoid sentences written like: we did this, then this, then this, etc.\newline It is strongly suggested to avoid expressions like `We think`, `We did`, etc...it is better to use impersonal expressions such as: `It is clear that`, `It is possible that`, `... something ... has been implemented/analyzed/etc.` (instead of `we did, we implemented, we analyzed`).\newline In the introduction you should give to the reader enough information to understand what is going to be explained in the remainder of the report (basically, expanding some concept you mentioned in the Abstract) without giving away too many information that would make the introduction too long and boring.\newline Feel free to organize the introduction in multiple sections and subsections, depending on how much content you want to put into this chapter.

%Remember that the introduction is needed to make the reader understand what kind of reading he or she will encounter. Be fluent and try not to confuse him or her.
%The introduction must ALWAYS end with the following formula: The remainder of the document is organized as follows. In Chapter 2, ...; in Chapter 3, ... so that the reader can choose which chapters are worth skipping according to the type of reading he or she has chosen.

In this chapter, it is presented an overview of the project, its objectives, and the scope of topics covered. The project revolves around the exploration and implementation of Complementary Metal-Oxide-Semiconductor Arbiter Physically Unclonable Functions (CamPUFs).
\section {Project Overview}
CamPUFs are a class of hardware-based security primitives that leverage the inherent physical variations within CMOS integrated circuits. These variations arise during the semiconductor manufacturing process, resulting in unique fingerprints for each individual chip. The primary objective of this project is to study, analyze, and implement CamPUFs to enhance hardware security and provide tamper-resistant cryptographic key generation.

\section {Objectives}
The project aims to achieve the following objectives:
\begin{itemize}
    \item \textbf{State of Art}: understanding CamPUFs regarding its theoretical foundation, mainly about properties, advantages and methods.
    

    \item \textbf{Desing and Implementation}: analyzing different CamPUFs architectures and choosing one of the different methods.
    
    
    \item \textbf{Demonstration}: testing with practical applications that the implementation works with a certain degree of accuracy.

\end{itemize}

\section {Document Organization}
This document is organized in the following way.
\begin{itemize}
\item \textbf{Chapter 1}: Introduction. This chapter.
\item \textbf{Chapter 2}: Background. In this chapter, all the theoretical concepts are discussed, starting with a summary about PUFs. Then, it is presented the principle behind the fingerprint extraction, that is the fixed-pattern noise, and, finally, it is explained how the fingerprint is exploited by way of the challenge-response pair.
\item \textbf{Chapter 3}: Implementation overview. It is explained the researches about PUFs, and the choosing of CamPUF as demonstrative project. Then, it is shown how the project has been developed and its high-level architecture.
\item \textbf{Chapter 4}: Implementation details. This chapter contains details useful to understand the code base, how to test and modify effectively the code and some programming choices.
\item \textbf{Chapter 5}: Results. Here there is a discussion about the dataset chosen, the results obtained, and the limits and future improvement of this project.
\item \textbf{Chapter 6}: Conclusions.
\end{itemize}
Moreover, it is provided in the Appendix [\ref{usermanual}] an user manual containing all the minimal information needed to replicate the project rapidly.
